% SolSeed Creed Commentary
% (c) by Brandon CS Sanders, Shelley Sanders, Benjamin Sibelman,
% Eric Saumur, and other contributing members of The SolSeed Movement
%
%
% SolSeed Creed Commentary is licensed under a
% Creative Commons Attribution-ShareAlike 3.0 Unported License.
%
% You should have received a copy of the license along with this
% work.  If not, see <http://creativecommons.org/licenses/by-sa/3.0/>.
\documentclass[ebook,12pt,openany,twoside]{memoir}
\usepackage[utf8]{inputenc}
\usepackage{setspace}
\usepackage{tocloft}
\usepackage{graphicx}
\usepackage{eso-pic}
\chapterstyle{bringhurst}
\openright

\setlength\stockheight {9.25in}% \stockheight=9.25in
\setlength\stockwidth  {6.25in}% \stockwidth=6.25in
\setlength\trimtop     {.125in}
\setlength\trimedge    {.125in}
\setlength{\paperwidth}{6.0in}
\setlength{\textwidth}{4.6in}
\setlength{\textheight}{6.5in}
\epigraphposition{flushright}
% \setlength\epigraphwidth{3.6in}

\newcommand{\tab}{\hspace*{2em}}

\newcommand{\imagefacingchapter}[1]{
  \cleartoverso
  \clearpage \null
  \thispagestyle{cleared}
  \AddToShipoutPictureBG*{% Add picture to current page
    \AtStockLowerLeft{% Add picture to lower-left corner of paper stock
      \includegraphics[keepaspectratio=true, height=\stockheight]{#1}
    }
  }
  \clearpage
}

\begin{document}


%\pagenumbering{}
% Set book title
\title{\textbf{SolSeed Creed Commentary}}
% Include Author name and Copyright holder name
\author{The SolSeed Movement}
\begin{titlingpage}
\maketitle
\end{titlingpage}


\cleartorecto
\thispagestyle{cleared}
\pagenumbering{arabic}

\pagestyle{plain}






\chapter{SolSeed Creed Version 2.1}

\begin{verse}
  Life is precious.\\
  \tab It has always been precious,\\
  \tab it will always be precious.

  Life exuberant\\
  bursting through boundaries\\
  must flower and spread\\
  \tab creating the conditions for more Life,\\
  \tab in an Upward Spiral of ever-growing possibilities.

  As you are alive, and I am alive,\\
  and in kinship with all other beings\\
  nourished by Sol,\\
  \tab we are SolSeed,--- \\
  \tab the body of all Life.

  The Destiny of SolSeed\\
  is to take root and flower amongst the stars ---\\
  \tab to give birth to a family of living worlds.

  As SolSeed's intelligent spark,\\
  \tab we are called to express the Arete of Life,\\
  \tab we are called to fulfill The Destiny.

  We who answer this call\\
  devote ourselves to Life.\\
  \tab We join together\\
  \tab in a community of practice\\
  \tab to align our actions\\
  \tab with our highest aspirations.

  Through art and science\\
  we awaken within ourselves and others\\
  \tab the cosmic religious feeling\\
  \tab that ignites wonder,\\
  \tab fosters compassion, and\\
  \tab inspires us to act.

  Our sacred duty is to\\
  embrace Passion,\\
  cultivate Empathy, and\\
  pursue Wisdom,\\
  \tab So that our being honors Life\\
  \tab and our striving advances The Destiny.

  Passion drives us.\\
  Without Passion,\\
  \tab Empathy and Wisdom are impotent.

  I pledge to stoke the fire in my belly,\\
  to compassionately care for my inner elephant ---\\
  to really be me, Happy in the Sun!

  Empathy is transcendent.\\
  Without Empathy,\\
  \tab Passion and Wisdom are evil.

  I pledge to love others as I love myself,\\
  to consider their needs as if they were my own ---\\
  to Grow Ours, not just Get Mine.

  Wisdom is effective.\\
  Without Wisdom,\\
  \tab Passion and Empathy are feeble and capricious.

  I pledge to seek truth by studying the world around me,\\
  to develop my character through regular practice,\\
  to cultivate sound instincts.

  Through Passion, Empathy, and Wisdom,\\
  we have come to know that:\\
  \tab We are Solseed\\
  \tab children of the Earth and Sun\\
  \tab awakened by starlight\\
  \tab growing\\
  \tab nurturing\\
  \tab discovering\\
  \tab We Bring Life!
\end{verse}






\imagefacingchapter{images/VenerableLife}
\chapter{Venerable Life}

\setlength\epigraphwidth{2.25in}
\epigraph{
  Life is precious.\\
  \tab It has always been precious,\\
  \tab it will always be precious.
}{}

\noindent Life puts atoms together in very interesting ways. One configuration
of atoms gives us an ant carrying a leaf many times larger than itself. From
another configuration we get an eagle slipping gracefully through the sky. Yet
another configuration becomes a person who feels, thinks, and loves.

Without Life, the energy from our sun simply bounces off the planet. Life
literally collects the energy Sol sends to earth, and stores it up for future
use. Plants take energy from Sol and convert it into denser, more storable
forms. The sugars and fats that our bodies burn are little pools of energy that
originally came from our sun.

Ants carrying, eagles soaring, people loving \ldots none of these would exist
without Life's awesome capacity for organizing matter.

Life is not perfect. This is obvious to anyone who has experienced cruelty or
misfortune. And yet, Life is the only game in town. Without Life the Earth
would be just another dead ball of rock.

Nothing needs to change for Life to become ``worthy''. It already is worthy.
Even if we had more wars, more hunger, and more systemic injustice, Life would
still still be precious. Always and forever, Life is what it is \ldots and it
is worthy of veneration!









\imagefacingchapter{images/UpwardSpiral}
\chapter{The Nature of Life}

\setlength\epigraphwidth{3.6in}
\epigraph{
  Life exuberant\\
  bursting through boundaries\\
  must flower and spread\\
  \tab creating the conditions for more Life,\\
  \tab in an Upward Spiral of ever-growing possibilities.
}{}

\noindent It is the nature of Life to flower and spread. During the Devonian
some 420 million years ago, Life burst out of the sea and took root on the
land. The period of adaptive radiation that followed drove an upward spiral of
innovation creating increasingly more complex expressions of Life.

It is awesome to realize that complex Life exists even though our universe is
strongly biased toward decay and collapse. We know from personal experience
that it is much easier to destroy than it is to create. Scientists call the
tendency for things to run down the ``second law of thermodynamics.'' The
second law can be summarized as ``energy always flows downhill toward less and
less usable forms.''

Life is the force that sometimes creates an upward eddy in the downward current
of destruction. The process of evolution that created the wonderful diversity
of species is a long slow steady climb. It is the story of the creation of
possibilities by countless individuals unwittingly making countless
contributions across countless generations.

Ants carrying, eagles soaring, people loving \ldots none of these would exist
without Life's slow and steady upward spiral of creation.






\imagefacingchapter{images/BodyOfAllLife}
\chapter{SolSeed: The Body of All Life}

\setlength\epigraphwidth{2.4in}
\epigraph{
  As you are alive, and I am alive,\\
  and in kinship with all other beings\\
  nourished by Sol,\\
  \tab we are SolSeed,--- \\
  \tab the body of all Life.
}{}

\noindent Groups of people may be organized into bodies. When we regard a
number of individuals as a single entity, we refer to the group as a ``body''.
We have legislative bodies, student bodies, administrative bodies, governing
bodies, and religious bodies. Bodies are in some way more than the sum of the
individuals. We expect a body of people to generate ideas and think about
things in a way that is different from what the individuals working alone would
come up with.

Groups of cells may also be organized into bodies.  The cells in your body don't
know who you are and don't care about you. And yet, by each cell doing its own
little thing in its own little context, this miraculous thing called
you emerges!

Just like our body is composed of cells that each do their own different unique
thing, so too there is a body of all life composed of living organisms of which
we are a part. This body of all Life has many names. We call it SolSeed. Sol
out of gratitude for the star at the center of our system that provides
practically all the energy that animates Life on Earth, and Seed to remind us
of the tremendous possibilities latent in the complex biosystem Life has
created by organizing the gift of energy that comes from Sol.






\imagefacingchapter{images/TakeRootAndFlower}
\chapter{Life Expressed!}

\setlength\epigraphwidth{3.2in}
\epigraph{
  The Destiny of SolSeed\\
  is to take root and flower amongst the stars ---\\
  \tab to give birth to a family of living worlds.
}{}

\noindent The evolution of Life in our solar system is far from complete. Earth
Life is now poised to burst off of our planet and take root amongst the other
bodies of our Solar system. The adaptive radiation that follows will again be a
period of tremendous innovation for Life. Life will begin to substantially
affect the rest of the Solar system. The changes Life will make will create
opportunities for even more Life.

Mother Earth and Father Sun are ready to start a family. Instead of our solar
system containing a single, lonely living world, there will be a family of
hundreds, thousands, and eventually millions of living worlds. Life will have a
tremendously enlarged playground over which to diversify.

In the very far future, Solar Life will spread beyond this solar system to take
root amongst the stars. As it does so, it may encounter Life from other sources
with which to commune.  Just think of the possibilities!







\imagefacingchapter{images/Called}
\chapter{Called}

\setlength\epigraphwidth{3.2in}
\epigraph{
  As SolSeed's intelligent spark,\\
  \tab we are called to express the Arete of Life,\\
  \tab we are called to fulfill The Destiny.
}{}

\noindent Our place in the body of all life is not one of special rights and
privileges, but rather one of great purpose and responsibility. In the words of
Carl Sagan,

\begin{quote}
\em
Our obligation to survive and flourish is owed not just to ourselves but also
to that Cosmos ancient and vast from which we spring.
\end{quote}

We are called not just to live -- not just to get by -- but to create a life
that brings honor to all the ancestors who came before us. We are called to
create a life that increases the opportunities of the generations that will
follow us.

Evolving humanity was expensive. It took eons to develop the biosphere and to
deposit vast stores of fossil fuels. In the decades of our infancy we caused
the extinction of myriad other species and consumed Life's densest reserves of
concentrated energy. It will take a lot of doing to become worthy of our price.


The extinctions we have caused and the fuels we have burned have purchased an
infrastructure that we can use in service to Life. The ultimate expression of
Life's flowering will come when we (the body of all life) transcend the bounds
of this planet and firmly take root throughout our solar system and the rest of
the galaxy. Humanity has an essential role to play in this ultimate expression
of the upward spiral of Life.

Humanity has a use, a role to play, a thing to do.




\imagefacingchapter{images/Dedicated}
\chapter{Dedicated}
\setlength\epigraphwidth{3.0in}
\epigraph{
  We who answer this call\\
  devote ourselves to Life.\\
  \tab We join together\\
  \tab in a community of practice\\
  \tab to align our actions\\
  \tab with our highest aspirations.
}{}

\noindent Many of life's choicest rewards come from being in relationship with
others. Our most important relationships gain special potency when they are
formed within the context of a larger shared purpose. Regular practice with a
community of like-minded indviduals keeps us rolling in our chosen direction
even when we lack inspiration and are overwhelmed by the concerns of everyday
life.

We join together in community in order to bring out the best in ourselves and
each other. We fan the flames of enthusiasm to nurture and encourage our
beginner selves. We call for a deepening of wisdom to uncover and let shine the
greatness in each of us. We direct the fire of passion down the line of
practical inquiry that brings with it understanding and faculty. We cultivate
habits that train us to instinctively react to even the most difficult
situations with responses that align with our highest aspirations.

Simply stated, Life is better when lived together.



\imagefacingchapter{images/Awakening}
\chapter{Awakening}
\setlength\epigraphwidth{3.0in}
\epigraph{
  Through art and the natural world\\
  we awaken within ourselves and others\\
  \tab the cosmic religious feeling\\
  \tab that ignites wonder,\\
  \tab fosters compassion, and\\
  \tab inspires us to proactively create.
}{}

There is a way of experiencing the world that is profoundly different than our
usual ego-centric mode. It feels like waking up to the true order of things.
The boundary between self and other dissolves as we directly experience the
universe as a single significant whole. We feel the futility of grasping after
things. In place of our habitual drive to react we find an empty space from
which to proactively initiate change.

This cosmic religious feeling is difficult to communicate from person to
person. It may be awakened or enlivened through a growing sense of awe or
uplifting emotion as we experience beauty, nobility, grandeur, or immensity.
These sublime experiences seem to often arise from the contemplation of art,
science, and nature. Individual and collective practice keeps us regularly
connected to these sources of inspiration.





\imagefacingchapter{images/StainedGlass}
% photos.com ... search StainedGlass
\chapter{Sacred Practice}
\setlength\epigraphwidth{3.0in}
\epigraph{
  Our sacred duty is to\\
  embrace Passion,\\
  cultivate Empathy, and\\
  pursue Wisdom,\\
  \tab So that our being honors Life\\
  \tab and our striving advances The Destiny.
}{}

\noindent The pursuit of scientific truth is one part of our religious purpose.
There is no logical proof that such a pursuit is good. There are no experiments
that we can do to demonstrate that an evidence based understanding of how the
world works is a good goal for our lives. And yet, we devote ourselves to this
pursuit as a higher calling. The importance of pursuing scientific truth is a
given in our lives that is as obvious as our own existence.

Greater even than our commitment to scientific truth is our devotion to expressing the full Arete of Life. We find tremendous uplifting significance in fully living up to our
potential, both individual and collective. As individuals we flower within our
local communities. Together we help Life take root and flower amongst the
stars. Fully expressing the Arete of Life brings honor to the unbroken thread
of ancestors who made our lives possible. The significance of this purpose is not rationally determined, but is basic to
our very nature. Similarly, the values from which this purpose springs have developed through our individual aesthetic reactions to direct individual experiences of profound meaning.

Our religion is not this purpose, but rather the activity of whole heartedly pursuing it. We strive to constantly
strengthen and promote the values and goals underlying our religious movement
... not because our religion demands it of us, but because the values and
purpose are the essential expression of our nature. The power of our religion comes not from its unique access to ``True values and
purposes'', but rather from its ability to provide us with the support to
effectively live the values we have chosen --- indeed --- the values that
have chosen us!



\imagefacingchapter{images/Elephant}
\chapter{Passion Drives Us}
\setlength\epigraphwidth{2.8in}
\epigraph{
  Passion drives us.\\
  Without Passion,\\
  \tab Empathy and Wisdom are impotent.
}{}

\noindent The love for an endeavor is a greater gift than is natural aptitude for it.
Natural aptitude alone will always be overtaken and surpassed by the ceaseless
practice of those who love an activity. In the words of de Saint-Exup\'{e}ry,
"if you would build a ship, don't tell people to gather wood and hammer nails,
instead teach them to long for the endless wild sea."

There was a time in my life when the most common thing you'd hear
from me was a sigh. Life was futile, tasteless and bland. It took all of my
reserves to meet the minimums. This ``sighing'' period of my life lasted for
months. Things got dramatically better when I began treatment for depression.
Suddenly I was full of ideas and brimming over with energy to try new things.

The contrast between these two adjacent periods of my life was startling. I,
the same person, could be more or less alive. I went from seeing Life as a
binary property where I was either alive or dead, to seeing Life as a
continuous property. An individual organism, a group of organisms, and indeed
the body of all life could be more or less alive from moment to moment.

We are only powerful when we are fully, passionately alive. We only do great
and beautiful things when we have enough gumption to get off the couch and take
action. If we lack the will to act, our best and wisest intentions come to nothing.

Passion inspires and motivates us. Passion is our source of power!




\imagefacingchapter{images/RidingElephant}
\chapter{Embrace Passion}
\setlength\epigraphwidth{3.0in}
\epigraph{
  I pledge to stoke the fire in my belly,\\
  to compassionately care for my inner elephant ---\\
  to really be me, Happy in the Sun!
}{}

\noindent The disparity between our goals and our actions leads us to be cruel
to ourselves. We use words like disappointment, lazy, and selfish to try and
whip ourselves into shape. The hurt and shame we inflict on ourselves fails to
bring about the changes we desire. We turn to addictive vices to escape our
self abuse.  The cycle of disappointment, abuse, and escapism continues.

Metaphors are powerful. They are how we make sense of the world and our place
in it. Perhaps no metaphor is more important than our metaphor for ourselves.
By adopting a more accurate self image we can foster compassion for ourselves
and finally discover the keys to realizing our highest aspirations.

In an age of cars, machines, and computers it is tempting to adopt a metaphor
of self that places our conscious mind in the role of operator and the rest of
us as the machine that is being operated. This metaphor is grossly inaccurate
and provides little insight about how to create what we seek to create in the
world.

One of the ancient truths that has been discovered over and over and commented
on by many different authors is the notion of the divided self. We are not
simply a conscious mind operating a piece of biological machinery. We are
actually a collection of different systems cooperating and competing to
accomplish their own individual goals. Not all of these systems are equally
intelligent, nor equally powerful.

In his book ''The Happiness Hypothesis'', social scientist Jonathan Haidt takes
a look at various metaphors for self that have received widespread attention
throughout the years. He then suggests that we adopt the model of our conscious
mind as a rider on top of the elephant of our unconscious mind and body.
Roughly speaking, the rider represents conscious reasoning and the elephant
represents instinct and emotional reactions. This simple metaphor does much to
explain day-to-day life.

The model of a rider and an elephant is particularly good at explaining the
differences between what we say our goals are and our actual behavior. The only
part of the rider-elephant combination that we have direct cognitive access to
is the rider. Any time we say something voluntarily, it is the rider who is
speaking (while the elephant is responsible for things like the reflexive yell
of pain on touching a hot stove). So when we talk about our goals, we give a
one-sided account that only incorporates what our rider wants.

The rider will inevitably lose a direct contest of wills against our elephant.
A skilled rider understands this and works sideways to create an environment in
which the pair can be successful. This means shaping the stimuli that the
elephant has access to, in order to limit exposure to undesirable situations
that take the elephant down a destructive path, and to increase exposure to
situations that lead to instinctive or emotional responses that motivate
positive action.

The single most important thing our rider can do is to understand what motivates our elephant and connect these motivators to our goals.  Such motivation does not come from clearly reasoned philosophical arguments.  Powerful motivation, the 'fire in our belly' so to speak, comes from stories and symbols that are experienced viscerally rather than cerebrally.  Evocative stories and symbols speak to our
unconscious minds and provide a short-cut pathway back to the well of
inspiration that keeps us motivated.

The path to a thriving, flourishing life runs through our elephant. The
beautiful life we seek depends upon our rider becoming a clever servant rather
than a cruel tyrant. 
Passion belongs to our elephant, it alone knows our unique
``happy in the sun!''







\chapter{Empathy is Transcendent}

\setlength\epigraphwidth{2.8in}
\epigraph{
  Empathy is transcendent.\\
  Without Empathy,\\
  \tab Passion and Wisdom are evil.
}{}

\noindent Empathy is the mechanism by which we understand and value others. Culture and society spring from Empathy (see Jonathan Haidt Righteous Mind
Crossing the Rubric chapter). Without Empathy the deep understanding and trust
so necessary for collective action cannot exist. It is Empathy that spurs us to
value the needs of others and consider how our actions might impact them.

It can be difficult to cultivate Empathy for strangers in other countries, for the entire body of all Life, and for future generations.  As we make our daily choices, we consider the needs of these abstract entities only as much as we identify with them in our guts.
And so we create religious images, metaphors, and rituals to give life to these abstract aspirations so that we can relate to them as immediately as we relate to a friend or family member.

Einstein as he developed the theory of relativity often imagined what
it would be like to ride alongside a beam of light. I invite you to do
something similar. Imagine yourself actually as a beam of light. A ray of
sunshine from our star whose name is Sol. As this sunbeam, you have several
different ways that you can think of yourself. On the one hand you are a
transient burst of photons that will soon be absorbed and cease to exist. On
the other hand you are the Sun itself. You are the provider of practically all
the energy that animates life on earth.

The ability to sometimes perceive ourselves not just as the ephemeral ray of
light, but as the sun itself can add a great deal of power and purpose to our
lives. This ability to identify with the larger whole is an essential
characteristic of spiritual enlightenment. It offers freedom from reactive
grasping and the option to proactively create our lives.





\chapter{Cultivate Empathy}

\setlength\epigraphwidth{2.8in}
\epigraph{
  I pledge to love others as I love myself,\\
  to consider their needs as if they were my own ---\\
  \tab to Grow Ours, not just Get Mine.
}{}

\noindent While it may be true that service of others provides happiness and life
satisfaction for the person who is doing the serving, the motivation for the
service is important. Something is missed when service to others is done to look good or even just
because we know that service leads to a happier life. Service brings the most peace and satisfaction when it springs from a depth of character where such service is simply the
right thing to do.

There is a trap of motivation that may be called spiritual narcissism. People
who seek after spiritual enlightenment, satisfaction, and peace as if they are
the ends themselves rather than the byproduct of a life of service find
themselves continually lacking peace and satisfaction. The more they focus on
peace and satisfaction, the bigger the hole becomes that they are trying to
fill.

Part of the paradox stems from the realization that the person we have the most
direct influence over is ourselves. This makes it seem like we ought to be
working directly on ourselves. Counterintuitively growth and development of ourselves is fostered
most directly by developing in ourselves a focus on others. This can be illustrated by
the attributes of a great conversationalist. The best conversationalists are
those who are actually interested in what the other person has to say. When the
other person pauses, the interested listener naturally rephrases what they
heard to verify they got it, or asks a question about something of particular
interest. Contrast this with the person whose main motivation is being heard.
The talker doesn't listen to the other person, instead they organize their own
thoughts so they can start talking in the first gap that is left.

Good religions demand that we live a life of character and
that we help others because it is "the right thing to do." As we empathize with the people we are seeking to help, we also cultivate appreciation for the gift they are giving to us by allowing us to participate in the transformative vulnerability of accepting assistance.  The more we humbly appreciate this gift, the more easily we will see the world through their
eyes, the more our actions in relationship with them will bring peace and satisfaction for everyone involved.

In the fiddler on the roof, Tevye the main character, offers a chunk of cheese to a young stranger who is passing through his town.  Perchik, the young man, refuses ... insisting "I have no money and I'm not a beggar."  Tevye responds, "Ah, take it. It's a blessing for me to give."  The genuiness of the sentiment wins Perchik over, and he accepts the food with a slightly pompous "Very well.  For your sake."  "Thank you. Thank you" replies Tevye as he hands him the food.  Throughout this interchange the empathy the older Tevye feels for young Perchik is clear.  Both men experience an increase in grace and dignity.




\chapter{Wisdom is Effective}

\setlength\epigraphwidth{3.8in}
\epigraph{
  Wisdom is effective.\\
  Without Wisdom,\\
  \tab Passion and Empathy are feeble and capricious.
}{}

Wisdom refers to the human ability to use knowledge and experience to develop
common sense and insight. Wisdom also refers to the human ability to discern or
judge what is true, right, or lasting.

Memory is a primitive piece of Wisdom. Memory is required for us to accurately recognize and characterize a situation.  Simulation is a more advanced piece of Wisdom. Simulation allows us to imagine how to create something in the future that is different than the current situation.

Even if we are passionately motivated to help others, if we do not understand the mechanisms of cause and effect our actions will be ineffective.  Worse yet, our actions may create unexpected outcomes that cause the situation to deteriorate rather than improve.

Wisdom is knowing the right thing to do in a given situation in order to bring about a good outcome.  The more practiced we are in responding to a given situation with Wisdom, the more reflexive and instinctive a good response becomes.  Practice is what turns received Wisdom into embodied Wisdom.

One aspect of wisdom is learning to change the patterns of our behavior
and establish better habits of thought and action; these habits conserve our
limited willpower by allowing us to automatically make choices in line with our
values. Another aspect is cultivating the intuition that helps us to decide on
the most positive and effective course of action in uncertain situations, where
our habits aren't enough to guide us. A third aspect is learning to balance and
synthesize the reasonable and emotional aspects of our minds, so that we avoid
making unreasonable choices based solely on emotion, but also accept and
celebrate the fact that without emotion, reason could never tell us the best
choice to make because no choice would ''feel'' better than any other. And
finally, wisdom is fundamentally about accepting reality as it is, not as we
would wish it to be (which is where science comes in handy), and only working
to change things after we've determined, to the best of our ability, where and
how we can actually make a positive difference in the world.





\chapter{Pursue Wisdom}
\setlength\epigraphwidth{3.8in}
\epigraph{
  I pledge to seek truth by studying the world around me,\\
  to develop my character through regular practice,\\
  to cultivate sound instincts.
}{}

\noindent Science is one manifestation of the
human capacity for Wisdom. When there is data available, we use it. We believe the scientific method is
currently our best mechanism to learn, codify, and share information about our
world and universe. The distillation of the scientific method is essentially
this:
\begin{itemize}[\textbullet]
\item Create a recipe that includes how and what you should expect to see

\item The community of the adequate can then perform the recipe and interpret
the results, honestly reporting whether the expectation was met or not

\item Recipes that are confirmed by others are elaborated on
\end{itemize}


But where do the recipes come from? Ideas for experiments and models with
predictive power are not generated by science. They are generated by Life's
creative spark. The same creative juices that stir one person to pen an opus
stir another person to create a theory to explain cosmic background radiation.
This creative force is mysterious. Where does it come from? We choose to see it
as an emergent property of life that occurs when the input from our senses
combines with memories and ideas stored in our brain in a unique way that
generates a sudden burst of thought, which synthesizes experience and memory in
a novel way that gives rise to a new idea. Again, we don't need to invoke
anything supernatural to describe this wondrous occurrence. Of course, it's
possible that there is some input into this process from a "higher realm"
beyond the everyday world of matter-energy-spacetime, but we don't have much to
say about that possibility because it's currently beyond the reach of
scientific study.

The community of the adequate is a concept we first learned about from Ken
Wilber. He makes the point that unless you have a super-collider, it is
difficult to play back the recipes that are currently being run at CERN - The
Large Hadron Collider. This requirement keeps the community of the adequate
quite small to test out the recipes they propose. Likewise, many mathematical
proofs involve concepts that are beyond the ken of most people. Evaluating
whether the proof is sound requires a level of expertise that comes only with
special aptitude and training. This too limits the size of the community of the
adequate. Mathematical recipes are no less real for being entirely within the
realm of mind. They are inner recipes rather than outer recipes.

So truth within many different realms can be communicated via the scientific
method. This is true within the objective physical realm as exemplified by the
"hard sciences" like chemistry, geology, neuroscience and the like. It is also
true within the realm of mind and can be seen in mathematics, logic, philosophy
and the "soft sciences" like psychology, sociology, and anthropology. The
scientific method even applies to the spiritual side of life to some degree,
and can be applied to meditation and other recipes for altering one's
perception of the ground of beingness. But this kind of research must be
approached with caution for a number of reasons, such as the fact that
subjective experience is very difficult to measure, and science works best with
quantitatively measurable results.

One of the degenerations that comes from science being most successfully
applied to the objective outer space is the collapse of inner and outer realms
and the collapse of spiritual with objective. An example of this can be seen in
experiments that try to show there is some sort of telepathic force field that
operates within and across life. These experiments confuse the essentially
subjective, spiritual feeling of "oneness with the world" with phenomena like
root systems or computer networks that objectively link things together. The
subjective feeling is important and valuable even if it doesn't correspond to
anything in the world around you, other than the impressions it makes on your
ordinary senses, which may combine with your learned knowledge of the objective
interdependence of people, species, and ecosystems within the global biosphere,
to trigger your spiritual experience.

When we say "religious practice," we don't mean what you probably think. It's
possible to write a scripture that doesn't resort to the supernatural, and
create rituals based around the sacredness of what we know to exist. In a way,
we're already doing this as a society. Scientific research papers comprise the
ever-changing scripture of a religion whose churches are laboratories and
university lecture halls. Its "priesthood," the researchers and science
professors, commands almost universal trust even though they all admit that
anything they say could turn out to be wrong.


\begin{quote}
\em
  Don't tell me what you believe, show me what you do and I will tell you what you believe. --civil rights movement quote
\end{quote}


% Last pages for ToC
%-------------------------------------------------------------------------------
\newpage
% Include dots between chapter name and page number
%Finally, include the ToC
\tableofcontents




\end{document}

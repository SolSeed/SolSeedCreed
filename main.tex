% SolSeed Creed Commentary
% (c) by Brandon CS Sanders, Shelley Sanders, Benjamin Sibelman,
% Eric Saumur, and other contributing members of The SolSeed Movement
%
%
% SolSeed Creed Commentary is licensed under a
% Creative Commons Attribution-ShareAlike 3.0 Unported License.
%
% You should have received a copy of the license along with this
% work.  If not, see <http://creativecommons.org/licenses/by-sa/3.0/>.
\documentclass[ebook,12pt,openany,twoside]{memoir}
\usepackage[utf8]{inputenc}
\usepackage{setspace}
\usepackage{tocloft}
\usepackage{graphicx}
\usepackage{eso-pic}
\chapterstyle{bringhurst}
\openright

\setlength\stockheight {9.25in}% \stockheight=9.25in
\setlength\stockwidth  {6.25in}% \stockwidth=6.25in
\setlength\trimtop     {.125in}
\setlength\trimedge    {.125in}
\setlength{\paperwidth}{6.0in}
\setlength{\textwidth}{4.6in}
\setlength{\textheight}{6.5in}
\epigraphposition{flushright}
% \setlength\epigraphwidth{3.6in}

\newcommand{\tab}{\hspace*{2em}}

\newcommand{\imagefacingchapter}[1]{
  \cleartoverso
  \clearpage \null
  \thispagestyle{cleared}
  \AddToShipoutPictureBG*{% Add picture to current page
    \AtStockLowerLeft{% Add picture to lower-left corner of paper stock
      \includegraphics[keepaspectratio=true, height=\stockheight]{#1}
    }
  }
  \clearpage
}

\begin{document}


%\pagenumbering{}
% Set book title
\title{\textbf{SolSeed Creed Commentary}}
% Include Author name and Copyright holder name
\author{The SolSeed Movement}
\begin{titlingpage}
\maketitle
\end{titlingpage}


\cleartorecto
\thispagestyle{cleared}
\pagenumbering{arabic}

\pagestyle{plain}

\chapter{SolSeed Creed Version 2.1}

\begin{verse}
  Life is precious.\\
  \tab It has always been precious,\\
  \tab it will always be precious.

  Life exuberant\\
  bursting through boundaries\\
  must flower and spread\\
  \tab creating the conditions for more Life,\\
  \tab in an Upward Spiral of ever-growing possibilities.

  As you are alive, and I am alive,\\
  and in kinship with all other beings\\
  nourished by Sol,\\
  \tab we are SolSeed,--- \\
  \tab the body of all Life.

  The Destiny of SolSeed\\
  is to take root and flower amongst the stars ---\\
  \tab to give birth to a family of living worlds.

  As SolSeed's intelligent spark,\\
  \tab we are called to express the Arete of Life,\\
  \tab we are called to fulfill The Destiny.

  We who answer this call\\
  devote ourselves to Life.\\
  \tab We join\\
  \tab together in a community of practice\\
  \tab to align our actions\\
  \tab with our highest aspirations.

  Through art and science\\
  we awaken within ourselves and others\\
  \tab the cosmic religious feeling\\
  \tab that ignites wonder,\\
  \tab fosters compassion, and\\
  \tab inspires us to act.

  Our sacred duty is to\\
  embrace Passion,\\
  cultivate Empathy, and\\
  pursue Wisdom,\\
  \tab So that our being honors Life\\
  \tab and our striving advances The Destiny.

  Passion drives us.\\
  Without Passion,\\
  \tab Empathy and Wisdom are impotent.

  I pledge to stoke the fire in my belly,\\
  to compassionately care for my inner elephant ---\\
  to really be me, Happy in the Sun!

  Empathy is transcendent.\\
  Without Empathy,\\
  \tab Passion and Wisdom are evil.

  I pledge to love others as I love myself,\\
  to consider their needs as if they were my own ---\\
  to Grow Ours, not just Get Mine.

  Wisdom is effective.\\
  Without Wisdom,\\
  \tab Passion and Empathy are feeble and capricious.

  I pledge to seek truth by studying the world around me,\\
  to develop my character through regular practice,\\
  to cultivate sound instincts.

  Through Passion, Empathy, and Wisdom,\\
  we have come to know that:\\
  \tab We are Solseed\\
  \tab children of the Earth and Sun\\
  \tab awakened by starlight\\
  \tab growing\\
  \tab nurturing\\
  \tab discovering\\
  \tab We Bring Life!
\end{verse}



\imagefacingchapter{images/LifeIsWorthyOfVeneration}
\chapter{Venerable Life}

\setlength\epigraphwidth{2.25in}
\epigraph{
  Life is precious.\\
  \tab It has always been precious,\\
  \tab it will always be precious.
}{}

\noindent Life puts atoms together in very interesting ways. One configuration
of atoms gives us an ant carrying a leaf many times larger than itself. From
another configuration we get an eagle slipping gracefully through the sky. Yet
another configuration becomes a person who feels, thinks, and loves.

Without Life, the energy from our sun simply bounces off the planet. Life
literally collects the energy Sol sends to earth, and stores it up for future
use. Plants take energy from Sol and convert it into denser, more storable
forms. The sugars and fats that our bodies burn are little pools of energy that
originally came from our sun.

Ants carrying, eagles soaring, people loving \ldots none of these would exist
without Life's awesome capacity for organizing matter.

Life is not perfect. This is obvious to anyone who has experienced cruelty or
misfortune. And yet, Life is the only game in town. Without Life the Earth
would be just another dead ball of rock.

Nothing needs to change for Life to become ``worthy''. It already is worthy.
Even if we had more wars, more hunger, and more systemic injustice, Life would
still still be precious. Always and forever, Life is what it is \ldots and it
is worthy of veneration!

\imagefacingchapter{images/UpwardSpiral}
\chapter{The Nature of Life}

\setlength\epigraphwidth{3.6in}
\epigraph{
  Life exuberant\\
  bursting through boundaries\\
  must flower and spread\\
  \tab creating the conditions for more Life,\\
  \tab in an Upward Spiral of ever-growing possibilities.
}{}

\noindent It is the nature of Life to flower and spread. During the Devonian
some 420 million years ago, Life burst out of the sea and took root on the
land. The period of adaptive radiation that followed drove an upward spiral of
innovation creating increasingly more complex expressions of Life.

It is awesome to realize that complex Life exists even though our universe is
strongly biased toward decay and collapse. We know from personal experience
that it is much easier to destroy than it is to create. Scientists call the
tendency for things to run down the ``second law of thermodynamics.'' The
second law can be summarized as ``energy always flows downhill toward less and
less usable forms.''

Life is the force that sometimes creates an upward eddy in the downward current
of destruction. The process of evolution that created the wonderful diversity
of species is a long slow steady climb. It is the story of the creation of
possibilities by countless individuals unwittingly making countless
contributions across countless generations.

Ants carrying, eagles soaring, people loving \ldots none of these would exist
without Life's slow and steady upward spiral of creation.

\imagefacingchapter{images/BodyOfAllLife-cropped}
\chapter{SolSeed: The Body of All Life}

\setlength\epigraphwidth{2.4in}
\epigraph{
  As you are alive, and I am alive,\\
  and in kinship with all other beings\\
  nourished by Sol,\\
  \tab we are SolSeed,--- \\
  \tab the body of all Life.
}{}

\noindent Groups of people may be organized into bodies. When we regard a
number of individuals as a single entity, we refer to the group as a ``body''.
We have legislative bodies, student bodies, administrative bodies, governing
bodies, and religious bodies. Bodies are in some way more than the sum of the
individuals. We expect a body of people to generate ideas and think about
things in a way that is different from what the individuals working alone would
come up with.

Groups of cells may also be organized into bodies.  The cells in your body don't
know who you are and don't care about you. And yet, by each cell doing its own
little thing in its own little context, this miraculous thing called
you emerges!

Just like our body is composed of cells that each do their own different unique
thing, so too there is a body of all life composed of living organisms of which
we are a part. This body of all Life has many names. We call it SolSeed. Sol
out of gratitude for the star at the center of our system that provides
practically all the energy that animates Life on Earth, and Seed to remind us
of the tremendous possibilities latent in the complex biosystem Life has
created by organizing the gift of energy that comes from Sol.

\imagefacingchapter{images/MotherEarthFatherSun}
\chapter{Life Expressed!}

\setlength\epigraphwidth{3.2in}
\epigraph{
  The Destiny of SolSeed\\
  is to take root and flower amongst the stars ---\\
  \tab to give birth to a family of living worlds.
}{}

\noindent The evolution of Life in our solar system is far from complete. Earth
Life is now poised to burst off of our planet and take root amongst the other
bodies of our Solar system. The adaptive radiation that follows will again be a
period of tremendous innovation for Life. Life will begin to substantially
affect the rest of the Solar system. The changes Life will make will create
opportunities for even more Life.

Mother Earth and Father Sun are ready to start a family. Instead of our solar
system containing a single, lonely living world, there will be a family of
hundreds, thousands, and eventually millions of living worlds. Life will have a
tremendously enlarged playground over which to diversify.

In the very far future, Solar Life will spread beyond this solar system to take
root amongst the stars. As it does so, it may encounter Life from other sources
with which to commune.  Just think of the possibilities!

\imagefacingchapter{images/TakeRootAmongstTheStars-comp}
\chapter{SolSeed's Intelligent Spark}

\setlength\epigraphwidth{2.7in}
\epigraph{
  As SolSeed's intelligent spark,\\
  \tab we are called to nurture Life,\\
  \tab we are called to fulfill The Destiny.
}{}

\noindent The history of life is a many branching tree and humanity is just one
branch on that tree. When inhabitants of the far future look back on humanity
as it is now, they will see not the pinnacle of evolution, but a transitional
species that diverged into new branches in the tree of life. Our place in the
body of all life is not one of special rights and privileges, but rather one of
purpose and responsibility. As Life's intelligent spark we are called to
nurture the body of all life -- we are called to ensure that it survives and
flourishes.

\begin{quote}
\em
Our obligation to survive and flourish is owed not just to ourselves but also
to that Cosmos ancient and vast from which we spring. --Carl Sagan
\end{quote}

We are called not just to live -- not just to get by -- but to create a life
that brings honor to all the ancestors who came before us. We are called to
create a life that increases the opportunities of the generations that will
follow us.

Evolving humanity was expensive. It took eons to develop the biosphere and to
deposit vast stores of fossil fuels. In the decades of our infancy we caused
the extinction of myriad other species and consumed Life's densest reserves of
concentrated energy. It will take a lot of doing to become worthy of our price.


The extinctions we have caused and the fuels we have burned have purchased an
infrastructure that we can use in service to Life. The ultimate expression of
Life's flowering will come when we (the body of all life) transcend the bounds
of this planet and firmly take root throughout our solar system and the rest of
the galaxy. Humanity has an essential role to play in this ultimate expression
of the upward spiral of Life.

Humanity has a use, a role to play, a thing to do.

\imagefacingchapter{images/StainedGlass}
% photos.com ... search StainedGlass
\chapter{Our Sacred Duty}
\setlength\epigraphwidth{3.0in}
\epigraph{
  Our sacred duty is to\\
  embrace Passion,\\
  cultivate Empathy, and\\
  pursue Wisdom,\\
  \tab So that our being honors SolSeed\\
  \tab and our striving advances The Destiny.
}{}

\noindent The pursuit of scientific truth is one part of our religious purpose.
There is no logical proof that such a pursuit is good. There are no experiments
that we can do to demonstrate that an evidence based understanding of how the
world works is a good goal for our lives. And yet, we devote ourselves to this
pursuit as a higher calling. The importance of pursuing scientific truth is a
given in our lives that is as obvious as our own existence.

Beyond just scientific truth, we are called to devote our lives to expressing
the Arete of Life. We find tremendous uplifting significance in fully living up
to our potential, both individually and collectively. Together we ought to help
Life take root and flower amongst the stars. Individually we ought to flower
ourselves within our local communities. Fully expressing the Arete of Life
brings honor to the unbroken thread of ancestors who made our lives possible.

The significance of this purpose is not rationally determined, but is basic to
our very nature. Similarly, the values from which this purpose springs are our
aethetic reactions to direct individual spiritual experiences.

Expressing the Arete of Life is our purpose. Our religion is not this purpose,
but rather the activity of whole heartedly pursuing it. We strive to constantly
strengthen and promote the values and goals underlying our religious movement
... not because our religion demands it of us, but because the values and
purpose are the essential expression of our nature.

To be effective in the long term pursuit of our values and purpose we benefit
from inspiration, clarity, and practice. When we are passionately inspired, we
work powerfully toward our goals. When our values are clear, we know how to
behave and what to pursue. Individual and collective practice keeps us
connected to regular sources of inspiration and clarity. These regular
practices create unconscious habits of being. These habits of being train us to
instinctively react to even the most difficult situations with responses that
align with our highest aspirations.

Evocative stories and symbols speak to our unconscious minds and provide a
short-cut pathway back to the well of inspiration that keeps us motivated.
Regular practice with a community of like-minded indviduals keeps us rolling in
our chosen direction even when we lack inspiration and the urgency of everday
life distract us from our super-personal purpose.

The power of our religion comes not from its unique access to ``True values and
purposes'', but rather from its ability to provide us with the support to
effectively live the values we have chosen --- indeed --- the values that
have chosen us!


\imagefacingchapter{images/Elephant}
\chapter{Passion Drives Us}
\setlength\epigraphwidth{2.8in}
\epigraph{
  Passion drives us.\\
  Without Passion,\\
  \tab Empathy and Wisdom are impotent.
}{}

\noindent There was a time in my life when the most common thing you'd hear
from me was a sigh. Life was futile, tasteless and bland. It took all of my
reserves to meet the minimums. This ``sighing'' period of my life lasted for
months. Things got dramatically better when I began treatment for depression.
Suddenly I was full of ideas and brimming over with energy to try new things.

The contrast between these two adjacent periods of my life was startling. I,
the same person, could be more or less alive. I went from seeing Life as a
binary property where I was either alive or dead, to seeing Life as a
continuous property. An individual organism, a group of organisms, and indeed
the body of all life could be more or less alive from moment to moment.

We are only powerful when we are fully, passionately alive. We only do great
and beautiful things when we have enough gumption to get off the couch and take
action. If we lack the will to act, our best and wisest intentions come to nothing.

Passion is individualistic; it drives us to care for and nurture ourselves, and
to work toward our own personal goals and dreams. Our Empathy projects our
Passion onto others, causing us to likewise care for and nurture them. Our love
for others can only be as strong and healthy as is our love for ourselves.
Without Passion, Empathy is hollow.

Passion inspires and motivates us. Passion is our source of power!

\imagefacingchapter{images/RidingElephant}
\chapter{Embrace Passion}
\setlength\epigraphwidth{3.0in}
\epigraph{
  I pledge to compassionately care for myself,\\
  to embrace my animal nature,\\
  to really be me, Happy in the Sun!
}{}

\noindent The disparity between our goals and our actions leads us to be cruel
to ourselves. We use words like disappointment, lazy, and selfish to try and
whip ourselves into shape. The hurt and shame we inflict on ourselves fails to
bring about the changes we desire. We turn to addictive vices to escape our
self abuse.  The cycle of disappointment, abuse, and escapism continues.

Metaphors are powerful. They are how we make sense of the world and our place
in it. Perhaps no metaphor is more important than our metaphor for ourselves.
By adopting a more accurate self image we can foster compassion for ourselves
and finally discover the keys to realizing our highest aspirations.

In an age of cars, machines, and computers it is tempting to adopt a metaphor
of self that places our conscious mind in the role of operator and the rest of
us as the machine that is being operated. This metaphor is grossly inaccurate
and provides little insight about how to create what we seek to create in the
world.

One of the ancient truths that has been discovered over and over and commented
on by many different authors is the notion of the divided self. We are not
simply a conscious mind operating a piece of biological machinery. We are
actually a collection of different systems cooperating and competing to
accomplish their own individual goals. Not all of these systems are equally
intelligent, nor equally powerful.

In his book ''The Happiness Hypothesis'', social scientist Jonathan Haidt takes
a look at various metaphors for self that have received widespread attention
throughout the years. He then suggests that we adopt the model of our conscious
mind as a rider on top of the elephant of our unconscious mind and body.
Roughly speaking, the rider represents conscious reasoning and the elephant
represents instinct and emotional reactions. This simple metaphor does much to
explain day-to-day life.

The model of a rider and an elephant is particularly good at explaining the
differences between what we say our goals are and our actual behavior. The only
part of the rider-elephant combination that we have direct cognitive access to
is the rider. Any time we say something voluntarily, it is the rider who is
speaking (while the elephant is responsible for things like the reflexive yell
of pain on touching a hot stove). So when we talk about our goals, we give a
one-sided account that only incorporates what our rider wants.

The rider will inevitably lose a direct contest of wills against our elephant.
A skilled rider understands this and works sideways to create an environment in
which the pair can be successful. This means shaping the stimuli that the
elephant has access to, in order to limit exposure to undesirable situations
that take the elephant down a destructive path, and to increase exposure to
situations that lead to instinctive or emotional responses that motivate
positive action.

Several years ago I happened upon a picture whose caption struck a chord in me
that resonates to this day. Our family was in the market to join a community
supported agriculture (CSA) program. As we browsed the websites for the various
CSAs in our area we happened across a picture of some sugar pea sprouts along
with the caption ``happy in the sun!''

``Happy in the sun!'' There was nothing complicated about the image or the
caption. The sprouts were simply being sprouts. And yet, there is something
profound about the phrase ``happy in the sun!'' and the way it was juxtaposed
with those sprouts. Those sprouts were living up to their potential. They were
in some sense ``fully alive,'' the very picture of thriving and flourishing.

The path to a thriving, flourishing life runs through our elephant. The
beautiful life we seek depends upon our rider becoming a clever servant rather
than a cruel tyrant. Passion belongs to our elephant, it alone knows our unique
``happy in the sun!''



% Last pages for ToC
%-------------------------------------------------------------------------------
\newpage
% Include dots between chapter name and page number
%Finally, include the ToC
\tableofcontents




\end{document}

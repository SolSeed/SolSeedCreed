% SolSeed Creed Commentary
% (c) by Brandon CS Sanders, Shelley Sanders, Benjamin Sibelman,
% Eric Saumur, and other contributing members of The SolSeed Movement
%
%
% SolSeed Creed Commentary is licensed under a
% Creative Commons Attribution-ShareAlike 3.0 Unported License.
%
% You should have received a copy of the license along with this
% work.  If not, see <http://creativecommons.org/licenses/by-sa/3.0/>.
\documentclass[ebook,12pt,openany,twoside]{memoir}
\usepackage[utf8]{inputenc}
\usepackage{setspace}
\usepackage{tocloft}
\usepackage{graphicx}
\usepackage{eso-pic}
\chapterstyle{bringhurst}
\openright

\setlength\stockheight {9.25in}% \stockheight=9.25in
\setlength\stockwidth  {6.25in}% \stockwidth=6.25in
\setlength\trimtop     {.125in}
\setlength\trimedge    {.125in}
\setlength{\paperwidth}{6.0in}
\setlength{\textwidth}{4.6in}
\setlength{\textheight}{6.5in}
\epigraphposition{flushright}
% \setlength\epigraphwidth{3.6in}

\newcommand{\tab}{\hspace*{2em}}

\newcommand{\imagefacingchapter}[1]{
  \cleartoverso
  \clearpage \null
  \thispagestyle{cleared}
  \AddToShipoutPictureBG*{% Add picture to current page
    \AtStockLowerLeft{% Add picture to lower-left corner of paper stock
      \includegraphics[keepaspectratio=true, height=\stockheight]{#1}
    }
  }
  \clearpage
}

\begin{document}


%\pagenumbering{}
% Set book title
\title{\tiny{The SolSeed Movement}\\\large{\textbf{We Believe}}}
% Include Author name and Copyright holder name
\author{\tiny{version 3.0.0}\\ \\ \\Brandon CS Sanders\\\small{and Kindred}}
\begin{titlingpage}
\maketitle
\end{titlingpage}



\pagestyle{plain}
\imagefacingchapter{images/VenerableLife}
\pagenumbering{arabic}
\chapter{Venerable Life}

\setlength\epigraphwidth{2.25in}
\epigraph{
  Life is precious.\\
  \tab It has always been precious.\\
  \tab It will always be precious.
}{}

\noindent Life puts atoms together in very interesting ways. One configuration
of atoms gives us an ant carrying a leaf many times larger than itself. From
another configuration we get an eagle slipping gracefully through the sky. Yet
another configuration becomes a person who feels, thinks, and loves.

Without Life, the energy from our sun simply bounces off the planet. Life
literally collects the energy Sol sends to earth, and stores it up for future
use. Plants take energy from Sol and convert it into denser, more storable
forms. The sugars and fats that our bodies burn are little pools of energy that
originally came from our sun.

Ants carrying, eagles soaring, people loving \ldots none of these would exist
without Life's awesome capacity for organizing matter.

Life is not perfect. This is obvious to anyone who has experienced cruelty or
misfortune. And yet, Life is the only game in town. Without Life the Earth
would be just another dead ball of rock.

Nothing needs to change for Life to become ``worthy''. It already is worthy.
Even if we had more wars, more hunger, and more systemic injustice, Life would
still be precious. Always and forever, Life is what it is \ldots and it
is worthy of veneration!









\imagefacingchapter{images/UpwardSpiral}
\chapter{The Nature of Life}

\setlength\epigraphwidth{3.6in}
\epigraph{
  Life exuberant\\
  bursting through boundaries\\
  to flower and spread\\
  \tab creates the conditions for more Life,\\
  \tab in an Upward Spiral of ever-growing possibilities.
}{}

\noindent It is the nature of Life to flower and spread. During the Devonian
420 million years ago, Life burst out of the sea and took root on land. The
adaptive radiation that followed drove an upward spiral of innovation,
dramatically increasing the diversity of Life.

Complex Life exists even though our universe is strongly biased toward decay
and collapse. Scientists call the tendency for things to run down the ``second
law of thermodynamics.'' Since the beginning of Life on our planet, effectively
every organism has died and every species has gone extinct (\textgreater
99.9\%).

Life is a process that sometimes creates an upward eddy in the downward current
of destruction dictated by the second law. Life created the diversity we see
today from countless individuals making countless contributions across countless
generations.

Arete is a Greek word that means living up to our potential and fulfilling our
purpose. When we say ``the Arete of Life'' we mean the complex whole of Life,
including the upward spiral of creation and the myriad deaths and extinctions
that give rise to that spiral.  We celebrate it all.

Ants carrying, eagles soaring, people loving \ldots none of these would exist
without Life's erratic yet tenacious upward spiral of creation.



\imagefacingchapter{images/BodyOfAllLife}
\chapter{SolSeed: The Body of All Life}

\setlength\epigraphwidth{2.4in}
\epigraph{
  As you are alive, and I am alive,\\
  and in kinship with all other beings\\
  nourished by Sol,\\
  \tab we are SolSeed,--- \\
  \tab the body of all Life.
}{}

\noindent Groups of people may be organized into bodies. When we regard a
number of individuals as a single entity, we refer to the group as a ``body''.
We have legislative bodies, student bodies, administrative bodies, governing
bodies, and religious bodies. Bodies are in some way more than the sum of the
individuals. We expect a body of people to generate ideas and think about
things in a way that is different from what the individuals working alone would
come up with.

Groups of cells may also be organized into bodies.  The cells in your body don't
know who you are and don't care about you. And yet, by each cell doing its own
little thing in its own little context, this miraculous thing called
you emerges!

Just like our body is composed of cells that each do their own different unique
thing, so too there is a body of all life composed of living organisms of which
we are a part. This body of all Life has many names. We call it SolSeed. Sol
out of gratitude for the star at the center of our system that provides
practically all the energy that animates Life on Earth, and Seed to remind us
of the tremendous possibilities latent in the complex biosystem Life has
created by organizing the gift of energy that comes from Sol.






\imagefacingchapter{images/TakeRootAndFlower}
\chapter{Life Expressed!}

\setlength\epigraphwidth{3.2in}
\epigraph{
  The Destiny of SolSeed\\
  is to take root and flower amongst the stars ---\\
  \tab to give birth to a family of living worlds.
}{}

\noindent The evolution of Life in our solar system is far from complete. Earth
Life is now poised to burst off of our planet and take root amongst the other
bodies of our Solar system. The adaptive radiation that follows will again be a
period of tremendous innovation for Life. Life will begin to substantially
affect the rest of the Solar system. The changes Life will make will create
opportunities for even more Life.

Mother Earth and Father Sun are ready to start a family. Instead of our solar
system containing a single, lonely living world, there will be a family of
hundreds, thousands, and eventually millions of living worlds. Life will have a
tremendously enlarged playground over which to diversify.

In the very far future, Solar Life will spread beyond this solar system to take
root amongst the stars. As it does so, it may encounter Life from other sources
with which to commune.  Just think of the possibilities!







\imagefacingchapter{images/Called}
\chapter{Called}

\setlength\epigraphwidth{3.2in}
\epigraph{
  As SolSeed's intelligent spark,\\
  \tab we are called to express the Arete of Life,\\
  \tab we are called to fulfill The Destiny.
}{}

\noindent Our place in the body of all life is not one of special rights and
privileges, but rather one of great purpose and responsibility. In the words of
Carl Sagan,

\begin{quote}
\em
Our obligation to survive and flourish is owed not just to ourselves but also
to that Cosmos ancient and vast from which we spring.
\end{quote}

We are called not just to live -- not just to get by -- but to create a life
that brings honor to all the ancestors who came before us. We are called to
create a life that increases the opportunities of the generations that will
follow us.

Evolving humanity was expensive. It took eons to develop the biosphere and to
deposit vast stores of fossil fuels. In the decades of our infancy we caused
the extinction of myriad other species and consumed Life's densest reserves of
concentrated energy. It will take a lot of doing to become worthy of our price.

The extinctions we have caused and the fuels we have burned have purchased an
infrastructure that we can use in service to Life. The ultimate expression of
Life's flowering will come when we (the body of all life) transcend the bounds
of this planet and firmly take root throughout our solar system and the rest of
the galaxy. Humanity has an essential role to play in this ultimate expression
of the upward spiral of Life.

Humanity has a use, a role to play, a thing to do.




\imagefacingchapter{images/Dedicated}
\chapter{Dedicated}
\setlength\epigraphwidth{2.2in}
\epigraph{
  We who answer this call\\
  dedicate ourselves to Life.\\
  \tab We join together\\
  \tab in a community of practice\\
  \tab to align our actions\\
  \tab with our highest aspirations.
}{}

\noindent Many of life's choicest rewards come from being in relationship with
others. Our most important relationships gain special potency when they are
formed within the context of a larger shared purpose. Regular practice with a
community of like-minded individuals keeps us rolling in our chosen direction
even when we lack inspiration and are overwhelmed by the concerns of everyday
life.

We join together in community in order to bring out the best in ourselves and
each other. We fan the flames of enthusiasm to nurture and encourage our
beginner selves. We call for a deepening of wisdom to uncover and let shine the
greatness in each of us. We direct the fire of passion down the line of
practical inquiry that brings with it understanding and faculty. We cultivate
habits that train us to instinctively react to even the most difficult
situations with responses that align with our highest aspirations.

Simply stated, Life is better when lived together.



\imagefacingchapter{images/Awakening}
\chapter{Awakening}
\setlength\epigraphwidth{2.6in}
\epigraph{
  Through art and the natural world\\
  we awaken within ourselves and others\\
  \tab the Cosmic Religious Feeling\\
  \tab that ignites wonder,\\
  \tab fosters compassion, and\\
  \tab inspires us to proactively create.
}{}

There is a way of experiencing the world that is profoundly different than our
usual ego-centric mode. It feels like waking up to the true order of things.
The boundary between self and other dissolves as we directly experience the
universe as a single significant whole. We feel the futility of grasping after
things. In place of our habitual drive to react we find an empty space from
which to proactively initiate change.

This cosmic religious feeling is difficult to communicate from person to
person. It may be awakened or enlivened through a growing sense of awe or
uplifting emotion as we experience beauty, nobility, grandeur, or immensity.
These sublime experiences seem to often arise from the contemplation of art,
science, and nature. Individual and collective practice keeps us regularly
connected to these sources of inspiration.





\imagefacingchapter{images/SacredPractice}
\chapter{Sacred Practice}
\setlength\epigraphwidth{3.0in}
\epigraph{
  Our sacred duty is to\\
  embrace Passion,\\
  cultivate Empathy, and\\
  pursue Wisdom,\\
  \tab So that our being honors Life\\
  \tab and our striving advances The Destiny.
}{}

\noindent The pursuit of scientific truth is a central part of our religious
practice. No logical proof or experiment can demonstrate that scientific truth
is good. And yet, we devote ourselves to science as to a higher calling.

Science is a uniquely human expression of the Arete of Life. Using science to
understand and manipulate the world for the benefit of Life brings honor to the
unbroken thread of ancestors who made our lives possible. We find tremendous
significance in living up to our potential in all ways, both individual and
collective.

If our purpose is to fully express the Arete of Life, then our religion is the
practice of whole heartedly pursuing this purpose. We strive to constantly
strengthen and promote the values and goals underlying our religious movement
... not because our religion demands it of us, but because the values and
purpose are the essential expression of our nature.

The power of our religion comes not from unique access to ``True'' values and
purposes, but from supporting us in living the values we have discovered in
ourselves. Our sacred practice helps us fulfill our individual duty to
flower in our communities and our species' duty to help Life take root
and flower amongst the stars.



\imagefacingchapter{images/PassionDrives}
\chapter{Passion Drives Us}
\setlength\epigraphwidth{2.8in}
\epigraph{
  Passion drives us.\\
  Without Passion,\\
  \tab Empathy and Wisdom are impotent.
}{}

\noindent The love for an endeavor is a greater gift than is natural aptitude for it.
Natural aptitude alone will always be overtaken and surpassed by the ceaseless
practice of those who love an activity. In the words of de Saint-Exup\'{e}ry,
``if you would build a ship, don't tell people to gather wood and hammer nails,
instead teach them to long for the endless wild sea.''

We are only powerful when we are fully, passionately alive. We only do great
and beautiful things when we have enough gumption to get off the couch and take
action. If we lack the will to act, our best and wisest intentions come to nothing.

In an age of cars, machines, and computers it is tempting to adopt a
metaphor of self that places our conscious mind in the role of operator and the
rest of us as the machine that is being operated. This metaphor is grossly
inaccurate and provides little insight about how to create what we seek to
create in the world. Instead we adopt the model of our conscious mind as a
rider on top of the elephant of our unconscious mind and body.

Motivations strong enough to be labeled as passions come almost entirely from
our elephants. We are attracted to some things and repelled from others not by
clearly reasoned philosophical arguments, but by the powerful aesthetic
reactions of our subconscious minds. The path to a thriving, flourishing life
runs through our elephants.  Their passion is our source of power!




\imagefacingchapter{images/EmbracePassion}
\chapter{Embrace Passion}
\setlength\epigraphwidth{3.4in}
\epigraph{
  I pledge to stoke the fire in my belly,\\
  to compassionately care for my inner elephant ---\\
  to really be me, Happy in the Sun!
}{}

\noindent The disparity between our goals and our actions leads us to be cruel
to ourselves. We use words like disappointment, lazy, and selfish to try and
whip ourselves into shape. The hurt and shame we inflict on ourselves fails to
bring about the changes we desire. We turn to addictive vices to escape our
self abuse.  The cycle of disappointment, abuse, and escapism continues.

Because we only have direct cognitive access to the rider, when we talk about
our goals we give a one-sided account that only incorporates what our rider
wants. The single most important thing our rider can do is to understand what
motivates our elephant and connect these motivators to our goals. The ``fire in
our belly'' so to speak, comes from stories and symbols that are experienced
viscerally rather than cerebrally.

The beautiful life we seek depends upon our rider becoming a clever servant
rather than a cruel tyrant. Passion belongs to our elephant, it alone knows our
unique ``happy in the sun!''






\imagefacingchapter{images/EmpathyIsTranscendant}
\chapter{Empathy is Transcendent}

\setlength\epigraphwidth{2.4in}
\epigraph{
  Empathy is transcendent.\\
  Without Empathy,\\
  \tab Passion and Wisdom are evil.
}{}

\noindent Empathy is the mechanism by which we understand and value others.
Culture and society spring from Empathy (see Jonathan Haidt Righteous Mind
Crossing the Rubric chapter). Without Empathy the deep understanding and trust
so necessary for collective action cannot exist. It is Empathy that spurs us to
value the needs of others and consider how our actions might impact them.

It can be difficult to cultivate Empathy for strangers in other countries, for
the entire body of all Life, and for future generations. As we make our daily
choices, we consider the needs of these abstract entities only as much as we
identify with them in our guts. And so we create religious images, metaphors,
and rituals to give life to these abstract aspirations so that we can relate to
them as immediately as we relate to a friend or family member.

The ability to sometimes perceive ourselves not just as an ephemeral individual
organism, but as the Body of all Life itself can add a great deal of power and
purpose to our lives. This ability to identify with the larger whole is an
essential characteristic of spiritual enlightenment. It offers freedom from
reactive grasping and the option to proactively create our lives.




\imagefacingchapter{images/CultivateEmpathy}
\chapter{Cultivate Empathy}
\setlength\epigraphwidth{3.4in}
\epigraph{
  I pledge to love others as I love myself,\\
  to consider their needs as if they were my own ---\\
  to Grow Ours, not just Get Mine.
}{}

\noindent While it may be true that service of others provides happiness and
life satisfaction for the person who is doing the serving, the motivation for
the service is important. Something is missed when service to others is done to
look good or even just because we know that service leads to a happier life.
Service brings the most peace and satisfaction when it springs from a depth of
character where such service is simply the right thing to do.

Good religions demand that we live a life of character and that we help others
because it is ``the right thing to do.'' As we empathize with the people we are
seeking to help, we also cultivate appreciation for the gift they are giving to
us by allowing us to participate in the transformative vulnerability of
accepting assistance.

In the fiddler on the roof, Tevye the main character, offers a chunk of cheese
to a young stranger who is passing through his town. Perchik, the young man,
refuses ... insisting ``I have no money and I'm not a beggar.'' Tevye responds,
``Ah, take it. It's a blessing for me to give.'' The genuineness of the sentiment
wins Perchik over, and he accepts the food with a slightly pompous ``Very well.
For your sake.'' ``Thank you. Thank you'' replies Tevye as he hands him the food.
Throughout this interchange the empathy the older Tevye feels for young Perchik
is clear. Both men experience an increase in grace and dignity.



\imagefacingchapter{images/WisdomIsEffective}
\chapter{Wisdom is Effective}

\setlength\epigraphwidth{3.5in}
\epigraph{
  Wisdom is effective.\\
  Without Wisdom,\\
  \tab Passion and Empathy are feeble and capricious.
}{}

Wisdom refers to the human ability to use knowledge and experience to develop
common sense and insight. Wisdom also refers to the human ability to discern or
judge what is true, right, or lasting.

Memory is a primitive piece of Wisdom. Memory is required for us to accurately
recognize and characterize a situation. Simulation is a more advanced piece of
Wisdom. Simulation allows us to imagine how to create something in the future
that is different than the current situation.

If we do not understand a situation our actions will be ineffective even if we
are passionately motivated to help others. Worse yet, our half-baked actions
may create unexpected outcomes that cause the situation to deteriorate rather
than to improve.

Wisdom is knowing a good action to take in a given situation in order to bring
about a good outcome. The more practiced we are in responding to a given
situation with Wisdom, the more reflexive and instinctive a good response
becomes.  Practice turns received Wisdom into embodied Wisdom.

When we are wise, we show good judgment even in difficult situations that most
people find overwhelming. With sufficient practice, we can quickly determine
what is truly important even amidst a cacophony of inputs.




\imagefacingchapter{images/PursueWisdom}
\chapter{Pursue Wisdom}
\setlength\epigraphwidth{3.8in}
\epigraph{
  I pledge to seek truth by studying the world around me,\\
  to develop my character through regular practice,\\
  to cultivate sound instincts.
}{}

\noindent The scientific method is currently our best mechanism to learn,
codify, and share information about our world and universe. The scientific
method is ideally suited for discovering and communicating truth about the
objective physical realm as exemplified by the ``hard sciences'' like
chemistry, geology, neuroscience and the like. With care we can also use the
scientific method to distill and communicate truth within other realms of
inquiry.

Within the realm of mind we can use the systematic techniques of the scientific
method to think about and communicate our results in mathematics, logic,
philosophy and the ``soft sciences'' like psychology, sociology, and
anthropology. Within the spiritual realm we can carefully evaluate forms of
meditation and other religious practices to determine which most reliably and
powerfully produce the fruits we are seeking.

In a very real sense, we become what we practice. Let us adopt again for the
moment the metaphor of our minds as a rider (our conscious mind) on an elephant
(our unconscious mind and body). We can develop wisdom by training the
rider/elephant combination to automatically react to situations in the ways we
desire. Regular, effective practice changes our patterns of behavior and
establishes better habits of thought and action.  These habits of good character
conserve our limited willpower by allowing us to automatically make choices in
line with our values without having to puzzle out the situational ethics of
every encounter.




\imagefacingchapter{images/WeBringLife}
\chapter{We Bring Life!}
\setlength\epigraphwidth{2.8in}
\epigraph{
  Through Passion, Empathy, and Wisdom,\\
  we have come to know that:\\
  \tab We are Solseed\\
  \tab children of the Earth and Sun\\
  \tab awakened by starlight\\
  \tab growing\\
  \tab nurturing\\
  \tab discovering\\
  \tab We Bring Life!
}{}

Life is not a binary property where we are either alive or dead. Life is a
continuous property such that an individual organism, a group of organisms, and
indeed the body of all life can be more or less alive from moment to moment. We
seek to bring more life to ourselves, our communities, our world, indeed even
unto the galaxy! Our religion succeeds in direct relation to how effective it
is in supporting us in bringing life at each of these scales of abstraction.


\end{document}
